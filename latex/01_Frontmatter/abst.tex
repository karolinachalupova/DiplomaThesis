\section*{Abstract}


Recent research shows that neural networks predict stock returns better than any other model. The networks' mathematically complicated nature is both their advantage, enabling to uncover complex patterns, and their curse, making them less readily interpretable, which obscures their strengths and weaknesses and complicates their usage. This thesis is one of the first attempts at overcoming this curse in the domain of stock returns prediction. Using some of the recently developed \textit{machine learning interpretability} methods, it explains the networks' superior return forecasts. This gives new answers to the long-standing question of which variables explain differences in stock returns and clarifies the unparalleled ability of networks to identify future winners and losers among the stocks in the market. Building on 50 years of asset pricing research, this thesis is likely the first to uncover whether neural networks support the economic mechanisms proposed by the literature. To a finance practitioner, the thesis offers the transparency of decomposing any prediction into its drivers, while maintaining a state-of-the-art profitability in terms of Sharpe ratio. Additionally, a novel metric is proposed that is particularly suited to interpret return-predicting networks in financial practice. This thesis offers a usable and economically explainable account of how machines make stock return predictions.

\bigskip

\begin{tabular}{lp{8.6cm}}
		\textbf{JEL Classification} & \JEL \\
		\textbf{Keywords} & \Keywords \\
 		& \\
		\textbf{Title} & \Bookname \\
 		\textbf{Author's e-mail} & \texttt{\href{mailto:\Email}{\Email}}\\
		\textbf{Supervisor's e-mail} & \texttt{\href{mailto:\EmailSup}{\EmailSup}}\\
		\textbf{Consultant's e-mail} & \texttt{\href{mailto:\EmailCon}{\EmailCon}}\\
\end{tabular}

\bigskip

\section*{Abstrakt}\label{abstract}

Nedávný výzkum ukazuje, \v{z}e neuronové sít\v{e} doká\v{z}ou p\v{r}edpovídat akciové výnosy lépe, ne\v{z} kterýkoli jiný model. Metematicky komplikovaná povaha sítí je zárove\v{n} jejich výhodou, umo\v{z}\v{n}ující odhalovat komplexní vzorce, a jejich prokletím, znesnad\v{n}ujícím jejich interpretaci, co\v{z} zaml\v{z}uje výhody a nevýhody sítí a komplikuje jejich u\v{z}ití. Tato práce je jedním z prvních pokus\r{u} toto prokletí p\v{r}ekonat. Za pou\v{z}ití nov\v{e} vyvinutých metod \textit{interpretovatelného strojového u\v{c}ení} objas\v{n}uje, jak sít\v{e} vytvá\v{r}ejí své vynikající p\v{r}edpov\v{e}di výnos\r{u}. Poskytuje tak nové odpov\v{e}di na starou otázku, které prom\v{e}nn\'{e} ur\v{c}ují rozd\'{i}ly v akciov\'{y}ch v\'{y}nosech, a vysv\v{e}tluje, co stoj\'{i} za bezkonkuren\v{c}n\'{i} schopnost\'{i} neuronov\'{y}ch s\'{i}t\'{i} identifikovat mezi akciemi na trhu budouc\'{i} v\'{i}t\v{e}ze a pora\v{z}en\'{e}. Tato práce je pravd\v{e}podobn\v{e} první, která zji\v{s}\v{t}uje, zda neuronové sít\v{e} podporují ekonomické mechanismy, které b\v{e}hem posledních 50 let p\v{r}inesl výzkum v oblasti oce\v{n}ování aktiv. Z hlediska aplikace pro finan\v{c}ní praxi práce nabízí transparentnost, kterou p\v{r}iná\v{s}í dekompozice ka\v{z}d\'{e} p\v{r}edpov\v{e}di na vlivy jednotliv\'{y}ch vstupn\'{i}ch prom\v{e}nn\'{y}ch; zárove\v{n} si práce zachovává Sharpe ratio na úrovni sou\v{c}asné v\v{e}dy. Nav\'{i}c je p\v{r}edstavena nov\'{a} metrika, kter\'{a} je zvl\'{a}\v{s}t\v{e} vhodn\'{a} pro interpretaci neuronov\'{y}ch s\'{i}t\'{i} ve finan\v{c}n\'{i} praxi. Tato pr\'{a}ce nab\'{i}z\'{i} aplikovateln\'{y} a ekonomicky vysv\v{e}tliteln\'{y} popis toho, jak stroje p\v{r}edpov\'{i}daj\'{i} v\'{y}nosy akci\'{i}.
 
\bigskip

\begin{tabular}{lp{7.7cm}}
		\textbf{Klasifikace JEL} & \JEL \\
		\textbf{Kl\'{i}\v{c}ov\'{a} slova} & \Klic \\
 		& \\
		\textbf{N\'{a}zev pr\'{a}ce} & \BooknameCZ \\
 		\textbf{E-mail autora} & \texttt{\href{mailto:\Email}{\Email}}\\
		\textbf{E-mail vedouc\'{i}ho pr\'{a}ce} & \texttt{\href{mailto:\EmailSup}{\EmailSup}}\\
		\textbf{E-mail konzultanta pr\'{a}ce} & \texttt{\href{mailto:\EmailCon}{\EmailCon}}\\
\end{tabular}

