\documentclass[12pt]{article}
 
\usepackage[margin=1.5in]{geometry} 
\usepackage{amsmath,amsthm,amssymb}

\pagenumbering{gobble}
 
\begin{document}

\title{%
	Mohou stroje vysv\v{e}tlit akciové výnosy? \\
	\large Abstrakt diplomové práce}
\author{Karolína Chalupová}
\maketitle

Nedávný výzkum ukazuje, \v{z}e neuronové sít\v{e} doká\v{z}ou p\v{r}edpovídat akciové výnosy lépe, ne\v{z} kterýkoli jiný model. Metematicky komplikovaná povaha sítí je zárove\v{n} jejich výhodou, umo\v{z}\v{n}ující odhalovat komplexní vzorce, a jejich prokletím, znesnad\v{n}ujícím jejich interpretaci, co\v{z} zaml\v{z}uje výhody a nevýhody sítí a komplikuje jejich u\v{z}ití. Tato práce je jedním z prvních pokus\r{u} toto prokletí p\v{r}ekonat. Za pou\v{z}ití nov\v{e} vyvinutých metod \textit{interpretovatelného strojového u\v{c}ení} objas\v{n}uje, jak sít\v{e} vytvá\v{r}ejí své vynikající p\v{r}edpov\v{e}di výnos\r{u}. Poskytuje tak nové odpov\v{e}di na starou otázku, které prom\v{e}nn\'{e} ur\v{c}ují rozd\'{i}ly v akciov\'{y}ch v\'{y}nosech, a vysv\v{e}tluje, co stoj\'{i} za bezkonkuren\v{c}n\'{i} schopnost\'{i} neuronov\'{y}ch s\'{i}t\'{i} identifikovat mezi akciemi na trhu budouc\'{i} v\'{i}t\v{e}ze a pora\v{z}en\'{e}. Tato práce je pravd\v{e}podobn\v{e} první, která zji\v{s}\v{t}uje, zda neuronové sít\v{e} podporují ekonomické mechanismy, které b\v{e}hem posledních 50 let p\v{r}inesl výzkum v oblasti oce\v{n}ování aktiv. Z hlediska aplikace pro finan\v{c}ní praxi práce nabízí transparentnost, kterou p\v{r}iná\v{s}í dekompozice ka\v{z}d\'{e} p\v{r}edpov\v{e}di na vlivy jednotliv\'{y}ch vstupn\'{i}ch prom\v{e}nn\'{y}ch; zárove\v{n} si práce zachovává Sharpe ratio na úrovni sou\v{c}asné v\v{e}dy. Nav\'{i}c je p\v{r}edstavena nov\'{a} metrika, kter\'{a} je zvl\'{a}\v{s}t\v{e} vhodn\'{a} pro interpretaci neuronov\'{y}ch s\'{i}t\'{i} ve finan\v{c}n\'{i} praxi. Tato pr\'{a}ce nab\'{i}z\'{i} aplikovateln\'{y} a ekonomicky vysv\v{e}tliteln\'{y} popis toho, jak stroje p\v{r}edpov\'{i}daj\'{i} v\'{y}nosy akci\'{i}.
 
\end{document}